\chapter{Advanced Data Representation}
\label{chap:advanced_data_representation}

\begin{figure}[ht]
	\hfill
	\begin{minipage}{0.5\textwidth}
		\centering
		\includegraphics{VTKTextbook-158}\\
		\includegraphics{VTKTextbook-157}
		\caption*{\texttt{Adaptive tessellation of higher-order cells.}}
	\end{minipage}
\end{figure}


\firstletter{T}his chapter examines advanced topics in data representation.
Topics include topological and geometric relationships and computational methods for cells and datasets.

\section{Coordinate Systems}
We will examine three different coordinate systems: the global, dataset, and structured coordinate systems.
Figure8–1 shows the relationship between the global and dataset coordinate systems, and depicts the structured coordinate system.

\subsection{Global Coordinate System}
The global coordinate system is a Cartesian, three-dimensional space. Each point is expressed as a triplet of values $(x,y,z)$ along the $x$, $y$, and $z$ axes.
This is the same system that was described in Chapter 3 (see \ref{sec:coordinate_systems}).
The global coordinate system is always used to specify dataset geometry (i.e., the point coordinates),and data attributes such as normals and vectors.
We will use the word ``position'' to indicate that we are using global coordinates.

\section{Interpolation Functions}
\label{sec:interpolation_functions}


\section{Searching}
\label{sec:searching}
Searching is an operation to find the cell containing a specified point \emph{p}, or to locate cells or points in a region surrounding \emph{p}.
Algorithms requiring this operation include streamline generation, where we need to find the starting location within a cell; probing, where the data values at a point are interpolated from the containing cell; or collision detection, where cells in a certain region must be evaluated for intersection.
Sometimes (e.g., image datasets), searching is a simple operation because of the regularity of data. However, in less structured data, the searching operation is more complex.

To find the cell containing \emph{p}, we can use the following naive search procedure.
Traverse all cells in the dataset, finding the one (if any) that contains \emph{p}.
To determine whether a cell contains a point, the cell interpolation functions are evaluated for the parametric coordinates $(r,s,t)$.
If these coordinates lie within the cell, then \emph{p} lies in the cell.
The basic assumption here is that cells do not overlap, so that at most a single cell contains the given point \emph{p}.
To determine cells or points lying in the region surrounding \emph{p}, we can traverse cells or points to see whether they lie within the region around \emph{p}.
For example, we can choose to define the region as a sphere centered at \emph{p}.
Then, if a point or the points composing a cell lie in the sphere, the point or cell is considered to be in the region surrounding \emph{p}.

\section{Putting It All Together}
In this section we will finish our earlier description of an implementation for unstructured data. We also define a high-level, abstract interface for cells and datasets. This interface allows us to implement the general (i.e., dataset specific) algorithms in the \emph{Visualization Toolkit}. We also describe
implementations for color scalars, searching and picking, and conclude with a series of examples to demonstrate some of these concepts.

\subsection{Picking}
\label{subsec:picking}

The Visualization Toolkit provides a variety of classes to perform actor (or vtkProp), point, cell, and
world point picking ( Figure8–38 ).

\subsection{Point Probe}
\label{subsec:point_probe}

\section{Chapter Summary}

Three important visualization coordinate systems are the world, dataset, and structured coordinate systems. The world coordinate system is an x--y--z Cartesian three-dimensional space. The dataset coordinate system consists of a cell id, subcell id, and parametric coordinates. The structured coordinate system consists of $(i,j,k)$ integer indices into a rectangular topological domain.

Visualization data is generally in discrete form. Interpolation functions are used to obtain data at points between the known data values. Interpolation functions vary depending on the particular cell type. The form of the interpolation functions are weighting values located at each of the cells points. The interpolations functions form the basis for conversion from dataset to global coordinates and vice versa. The interpolation functions also are used to compute data derivatives.

Topological operators provide information about the topology of a cell or dataset. Obtaining neighboring cells to a particular cell is an important visualization operation. This operation can be used to determine whether cell boundaries are on the boundary of a dataset or to traverse datasets on a cell-by-cell basis.

Because of the inherent regularity of image datasets, operations can be efficiently implemented compared to other dataset types. These operations include coordinate transformation, derivative computation, topological query, and searching.

\section{Bibliographic Notes}

Interpolation functions are employed in a number of numerical techniques. The finite element method, in particular, depends on interpolation functions. If you want more information about interpolation functions refer to the finite element references suggested below \cite{Cook89} \cite{Gallagher75} \cite{Zienkiewicz87}. These texts also discuss derivative computation in the context of interpolation functions.

Visualizing higher-order datasets is an oepn research issue. While \cite{Schroeder06} describes one approach, methods based on GPU programs are emerging. Other approaches include tailored algorithms for a particular cell type.

Basic topology references are available from a number of sources. Two good descriptions of topological data structures are available from Weiler \cite{Weiler86} \cite{Weiler88} and Baumgart \cite{Baumgart74}. Weiler describes the radial-edge structure. This data structure can represent manifold and nonmanifold geometry. The winged-edge structure described by Baumgart is widely known. It is used to represent manifold geometry. Shephard \cite{Shephard88} describes general finite element data structures — these are similar to visualization structures but with extra information related to analysis and geometric modelling.

There are extensive references regarding spatial search structures. Samet \cite{Samet90} provides a general overview of some. Octrees were originally developed by Meagher \cite{Meagher82} for 3D imaging. See \cite{Williams83}, \cite{Bentley75}, and \cite{Quinlan94} for information about MIP maps, kd-trees, and binary sphere trees, respectively.

\printbibliography


\section{Exercises}