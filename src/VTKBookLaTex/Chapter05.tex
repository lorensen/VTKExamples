\chapter{Basic Data Representation}
\label{chap:basic_data_representation}

\begin{figure}[ht]
	\hfill
	\begin{minipage}{0.5\textwidth}
		\centering
		\includegraphics{VTKTextbook-48}\\
		\caption*{\texttt{Compatible tessellations.}}
	\end{minipage}
\end{figure}

\firstletter{I}n Chapter 4 we developed a pragmatic definition of the visualization process: mapping information into graphics primitives.
We saw how this mapping proceeds through one or more steps, each step transforming data from one form, or data representation, into another.
In this chapter we examine common data forms for visualization.
The goal is to familiarize you with these forms, so that you can visualize your own data using the tools and techniques provided in this text.

\section{Introduction}
To design representational schemes for data we need to know something about the data we might
encounter. We also need to keep in mind design goals, so that we can design efficient data structures
and access methods. The next two sections address these issues.

\subsection{Characterizing Visualization Data}
Since our aim is to visualize data, clearly we need to know something about the character of the
data. This knowledge will help us create useful data models and powerful visualization systems.
